% File: rub-archaeologie-example-de.tex
% Copyright (C) 2024 by Joran Schneyer <joran.schneyer@ruhr-uni-bochum.de>
%
% This work may be distributed and/or modified under the
% conditions of the LaTeX Project Public License, either version 1.3c
% of this license or (at your option) any later version.
% The latest version of this license is in
%   https://www.latex-project.org/lppl.txt
% and version 1.3c or later is part of all distributions of LaTeX
% version 2008 or later.
%
% This work has the LPPL maintenance status `maintained'.
% 
% The Current Maintainer of this work is Joran Schneyer <joran.schneyer@ruhr-uni-bochum.de>.
%
% This work consists of the files rub-archaeologie.dtx
%                                 rub-archaeologie.ins
%                                 rub-archaeologie-example-de.tex
%                                 rub-archaeologie-example-en.tex
%           and the derived file  rub-archaeologie.cls

% We start by specifying the document class
\documentclass{rub-archaeologie}

% We use the "babel" package to set the language of the document to english
% This is used in many places e.g. for automated word-breaks, language of default section titles such as "table of contents" etc.
\usepackage[english]{babel}
% We also need to tell the "hyperref" package to use this language as it was already loaded by the rub-archaeologie class.
\hypersetup{english=true}

% We then add information for the title page
\title{Example usage of the \textsf{rub-archaeologie} class}
\author{Joran Schneyer}

% Here, we start the actual document
\begin{document}
    % First, we print the title page
    \maketitle

    % Next, we show the (automatically generated) table of contents
    \tableofcontents
    % and keep the rest of the page empty continuing on the next one
    \clearpage

    % We start the main content with an introduction
    \section{Introduction}
    The \textsf{rub-archaeologie} class aims to implement the guidelines on scientific writing of the Institute for Archaeological Studies at Ruhr University Bochum. This example document serves as a template for users of the class to start their own projects and/or as a reference that complements the full documentation\footnote{The most recent version of the full documentation can be found e.g. on Github: \url{https://github.com/BizarrePenguin/rub-archaeologie-latex/releases/latest/download/rub-archaeologie.pdf}.} by providing examples for typical usage while not covering all features and possibilities.

    Alongside this document in english there is also a variant in german, called \texttt{rub-archaeologie-example-de.tex}\footnote{\url{https://github.com/BizarrePenguin/rub-archaeologie-latex/releases/latest/download/rub-archaeologie-example-de.tex}} as well as the compiled pdf version \texttt{rub-archaeologie-example-de.pdf}\footnote{\url{https://github.com/BizarrePenguin/rub-archaeologie-latex/releases/latest/download/rub-archaeologie-example-de.pdf}}.
\end{document}