% \iffalse meta-comment
%
% File: rub-archaeologie.dtx
% Copyright (C) 2024 by Joran Schneyer <joran.schneyer@ruhr-uni-bochum.de>
%
% This work may be distributed and/or modified under the
% conditions of the LaTeX Project Public License, either version 1.3c
% of this license or (at your option) any later version.
% The latest version of this license is in
%   https://www.latex-project.org/lppl.txt
% and version 1.3c or later is part of all distributions of LaTeX
% version 2008 or later.
%
% This work has the LPPL maintenance status `maintained'.
% 
% The Current Maintainer of this work is Joran Schneyer <joran.schneyer@ruhr-uni-bochum.de>.
%
% This work consists of the files rub-archaeologie.dtx
%                                 rub-archaeologie.ins
%           and the derived files rub-archaeologie.cls
%
% \fi

% \iffalse
%<*driver>
\ProvidesFile{rub-archaeologie.dtx}
%</driver>
%<class>\NeedsTeXFormat{LaTeX2e}[2022-06-01]
%<class>\ProvidesClass{rub-archaeologie}
%<*class>
    [2025-01-30 v0.0.0 RUB Archaeology class]
%</class>
%<*driver>
\documentclass{ltxdoc}

%^^A Load packages needed for the documentation
\usepackage{dtxdescribe}
\fvset{commandchars={|()}}

\EnableCrossrefs
\CodelineIndex
\RecordChanges
\begin{document}
    \DocInput{\jobname.dtx}
    \PrintChanges
    \PrintIndex
\end{document}
%</driver>
% \fi
%
%^^A Document general changes here
% \changes{unreleased}{unreleased}{Initial version}
%
% \GetFileInfo{\jobname.dtx}
%
% \DoNotIndex{\newcommand,\newenvironment}
% \DoNotIndex{\begin,\end}
% \DoNotIndex{\fi}
% \DoNotIndex{\section,\subsection,\subsubsection}
% \DoNotIndex{\title,\author,\tableofcontents}
% \DoNotIndex{\LaTeX,\verb}
% \DoNotIndex{\texorpdfstring}
%
% \DoNotIndex{\@rubarchaeo@biblatexOptions}
% \DoNotIndex{\@rubarchaeo@parskiptrue}
%
%^^A define helper commands for consistent typesetting in the documentation
% \DeclareDocumentCommand\email{m}{\href{mailto:#1}{\nolinkurl{#1}}}
%
% \title{The \pkg{\jobname} class^^A
%   \thanks{This document corresponds to \pkg{\jobname}~\fileversion,
%     dated \filedate.}}
% \author{\copyright{} Joran Schneyer^^A
%   \thanks{Released under the LaTeX Project Public License v1.3c or later.^^A
%   \\ See \url{https://www.latex-project.org/lppl.txt}}^^A
%   \\ \email{joran.schneyer@ruhr-uni-bochum.de}}
% \date{\filedate}
%
% \maketitle
% \tableofcontents
%
% \section{Introduction}\label{sec:introduction}
%
% This \LaTeX{} class aims to implement the guidelines on scientific writing of the Institute of Archaeological Studies at Ruhr University Bochum.^^A
% \footnote{Guidelines version 3.0 (November 2024) \url{https://www.archwiss.ruhr-uni-bochum.de/aw/mam/wissenschaftlicher_reader_studierende_standnov2024.pdf}}
%
% Note, that at this point this is not an official class made by anyone at the institute but rather a free-time hobby project of me, Joran, who knows \LaTeX{} from studying Electrical Engineering and just wants to help out some friends studying archaeology.
%
% You can find the latest releases and the development of this project at GitHub: \url{https://github.com/BizarrePenguin/rub-archeaologie-latex}
%
% \section{Usage}\label{sec:usage}
%
% To use this class, simply specify it as the document class.^^A
% \begin{sourceverb}
% \documentclass{rub-archaeologie}
% \end{sourceverb}
%
% \StopEventually{}
%
% \clearpage
% \appendix
%
% \section{Implementation}\label{sec:implementation}
%
% \iffalse
%<*class>
% \fi
%
% \subsection{Class options}\label{sec:implementation:class-options}
% \iffalse
%% Class options
% \fi
% The class options are defined as keyval options for great flexibility.
% They toggle some of the class features or customize their behavior.
%
% \begin{option}{biblatex}
% The \optn{biblatex} option stores its content in a macro until it is later passed along to the \pkg{biblatex} package after specifying the default options (see \autoref{sec:implementation:package-loading:bibliography}).
% Therefore, options provided with this key overwrite the ones set per default by this class.
%    \begin{macrocode}
\DeclareKeys[rubarchaeo]{
    biblatex.store = \@rubarchaeo@biblatexOptions,
    biblatex.usage = load
}
%    \end{macrocode}
% \end{option}
%
% \begin{option}{hyperref}
% The \optn{hyperref} option passes its content along immediately, to be used as options when the \pkg{hyperref} package is loaded.
% This will \textbf{not} overwrite default options set by this class as they are set after loading the package using \verb|\hypersetup|\marg{options} (see also \autoref{sec:implementation:package-loading:others}).
%    \begin{macrocode}
\DeclareKeys[rubarchaeo]{
    hyperref.code  = \PassOptionsToPackage{#1}{hyperref},
    hyperref.usage = load
}
%    \end{macrocode}
% \end{option}
%
% \paragraph{Paragraph indentation}
% First we define a TeX switch which is later used (see \autoref{sec:implementation:package-loading}) to check wether to load the \pkg{parskip} package to remove the indentation at the start of paragraphs.
% \iffalse
%% TeX switch to decide wether to load the parskip package
% \fi
%    \begin{macrocode}
\newif\if@rubarchaeo@parskip
%    \end{macrocode}
% By default we set the switch to true, so the parskip package is normally loaded when using this class.
%    \begin{macrocode}
\@rubarchaeo@parskiptrue
%    \end{macrocode}
% \begin{option}{parskip}
% Then we declare the key so the switch can be turned on and off by the user in the class options.
%    \begin{macrocode}
\DeclareKeys[rubarchaeo]{
    parskip.if      = @rubarchaeo@parskip,
    parskip.usage   = load,
%    \end{macrocode}
% \end{option}
% \begin{option}{noparskip}
% Finally we define the complementary key for easier disabling of the parskip feature.
%    \begin{macrocode}
    noparskip.ifnot = @rubarchaeo@parskip,
    noparskip.usage = load
}
%    \end{macrocode}
% \end{option}
%
% \paragraph{Process options}
% After defining the class options it is necessary to process them too in order to actually make use of them.
%    \begin{macrocode}
\ProcessKeyOptions[rubarchaeo]
%    \end{macrocode}
%
% \subsection{Base class}\label{sec:implementation:base-class}
% \DescribeClass{article}
% The \pkg{\jobname} class is based on the \pkg{article} class.
% When loading the class we specify \texttt{12pt} as the base font size, as required by the guidelines.
% \iffalse
%% Load base class with 12pt base font size
% \fi
%    \begin{macrocode}
\LoadClass[12pt]{article}
%    \end{macrocode}
%
% \subsection{Loading packages}\label{sec:implementation:package-loading}
%
% \subsubsection{Bibliography}\label{sec:implementation:package-loading:bibliography}
%
% \DescribePackage{biblatex}
% To support bibliography facilities out of the box, the \pkg{biblatex} package is loaded. To customize the behavior, a few options are passed to the package.
% \iffalse
%% biblatex options
% \fi
%    \begin{macrocode}
\PassOptionsToPackage{
%    \end{macrocode}
%
% \DescribeOption[biblatex]{backend}
% Defines the backend program. To use all features of \pkg{biblatex}, the \prog{biber} backend must be used.
%    \begin{macrocode}
    backend=biber,
%    \end{macrocode}
%
% \DescribeOption[biblatex]{autocite}
% Defines the behavior of the \cs{autocite} command, which in this case will behave like the \cs{footcite} command and will print citations in the footnotes instead of directly in the text.
%    \begin{macrocode}
    autocite=footnote,
%    \end{macrocode}
%
% \DescribeOption[biblatex]{autopunct}
% Controls whether the citation commands scan ahead for punctuation marks and move the citation after the punctuation.
%    \begin{macrocode}
    autopunct=true
}{biblatex}
%    \end{macrocode}
%
% \DescribePackage{archaeologie}
% By default we use the biblatex style of the \pkg{archaeologie} package\footnote{The package and its documentation can be found here: \url{https://ctan.org/pkg/archaeologie}}.
%    \begin{macrocode}
\PassOptionsToPackage{
%    \end{macrocode}
%
% \DescribeOption[biblatex]{style}
% The \optn{archaelogie} style provided by the \pkg{archaeologie} package claims to implement the citation style of the german archaeological institute (Deutsches Archäologisches Institut, short: DAI).
%    \begin{macrocode}
    style=archaeologie,
%    \end{macrocode}
%
% Now we end the cs{PassOptionsToPackage} command:
%    \begin{macrocode}
}{biblatex}
%    \end{macrocode}
%
% After specifying the default options, we pass along the ones that might have been set in the class options.
% \iffalse
%% pass biblatex class options along
% \fi
%    \begin{macrocode}
\PassOptionsToPackage{\@rubarchaeo@biblatexOptions}{biblatex}
%    \end{macrocode}
%
% And finally the \pkg{biblatex} package is loaded.
% \iffalse
%% load biblatex
% \fi
%    \begin{macrocode}
\RequirePackage{biblatex}
%    \end{macrocode}
%
%
% \subsubsection{Document format}
%
% \DescribePackage{parskip}
% To avoid indentation at the start of paragraphs, the \pkg{parskip} package is loaded if the corresponding switch is set to true. See also \autoref{sec:implementation:class-options}.
% \iffalse
%% Avoid paragraph indentation
% \fi
%    \begin{macrocode}
\if@rubarchaeo@parskip
    \RequirePackage{parskip}
\fi
%    \end{macrocode}
%
% \subsubsection{Other useful packages}\label{sec:implementation:package-loading:others}
%
\DescribePackage{hyperref}
% Makes links and references clickable.
% \iffalse
%% Hyperref
% \fi
%    \begin{macrocode}
\RequirePackage{hyperref}
%    \end{macrocode}
% By default, we configure it to not highlight clickable elements.
%    \begin{macrocode}
\hypersetup{hidelinks}
%    \end{macrocode}
%
% \iffalse

%</class>
% \fi
%
% \Finale
