% File: rub-archaeologie-example-de.tex
% Copyright (C) 2024 by Joran Schneyer <joran.schneyer@ruhr-uni-bochum.de>
%
% This work may be distributed and/or modified under the
% conditions of the LaTeX Project Public License, either version 1.3c
% of this license or (at your option) any later version.
% The latest version of this license is in
%   https://www.latex-project.org/lppl.txt
% and version 1.3c or later is part of all distributions of LaTeX
% version 2008 or later.
%
% This work has the LPPL maintenance status `maintained'.
% 
% The Current Maintainer of this work is Joran Schneyer <joran.schneyer@ruhr-uni-bochum.de>.
%
% This work consists of the files rub-archaeologie.dtx
%                                 rub-archaeologie.ins
%                                 rub-archaeologie-example-de.tex
%                                 rub-archaeologie-example-en.tex
%           and the derived file  rub-archaeologie.cls

% Wir beginnen damit die Klasse festzulegen
\documentclass{rub-archaeologie}

% Wir nutzen das Paket babel um die Sprache einzustellen. Dies beeinflusst Dinge wie automatische Silbentrennung, automatische Überschriften wie z.B. für das Inhaltsverzeichnis, Datumsformate etc.
% "ngerman" steht für die neue deutsche Rechtschreibung (wichtig z.B. für die korrekte automatische Silbentrennung)
\usepackage[ngerman]{babel}
% Dem Paket "hyperref" muss die Sprache auch noch mitgeteilt werden, da es bereits oben bei \documentclass von der rub-archaeologie Klasse geladen wurde und noch nicht weiß, dass wir babel nutzen.
\hypersetup{ngerman=true}

% Dann fügen wir Informationen für die Titelseite hinzu
\title{Beispiel für die Nutzung der \textsf{rub-archaeologie} Klasse}
\author{Joran Schneyer}

% Hier starten wir das eigentliche Dokument
\begin{document}
    % Wir zeigen zunächst die Titelseite
    \maketitle

    % Es folgt das (automatische) Inhaltsverzeichnis
    \tableofcontents
    % und wir lassen den Rest der Seite leer und machen auf der nächsten Seite weiter
    \clearpage

    % Dann fangen wir mit der Einleitung an
    \section{Einleitung}
    Die \textsf{rub-archaeologie} Klasse versucht die Richtlinien für wissenschaftliches Arbeiten des Instituts für Archäologische Wissenschaften an der Ruhr Universität Bochum in \LaTeX zu implementieren.
    Dieses Beispieldokument dient als Vorlage für Nutzer*Innen der Klasse um darauf basierend ihre eigenen Projekte umzusetzen. Es dient darüber hinaus als Referenz, welche die vollständige Dokumentation\footnote{Die vollständige Dokumentation ist auf Englisch verfasst und die aktuellste Version ist z.B. auf Github verfügbar unter: \url{https://github.com/BizarrePenguin/rub-archaeologie-latex/releases/latest/download/rub-archaeologie.pdf}.} komplementiert und diese um typische Beispiele der Nutzung erweitert, ohne alle Funktionen und Situationen abzudecken.

    Neben diesem Beispieldokument auf Deutsch gibt es auch eines auf Englisch mit dem Dateinamen \texttt{rub-archaeologie-example-en.tex}\footnote{\url{https://github.com/BizarrePenguin/rub-archaeologie-latex/releases/latest/download/rub-archaeologie-example-en.tex}} und der entsprechenden pdf Version als \texttt{rub-archaeologie-example-en.pdf}\footnote{\url{https://github.com/BizarrePenguin/rub-archaeologie-latex/releases/latest/download/rub-archaeologie-example-en.pdf}}.
\end{document}